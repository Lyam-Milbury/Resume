%% If you want to use \orcid or the
%% academicons icons, add "academicons"
%% to the \documentclass options. 
%% Then compile with XeLaTeX or LuaLaTeX.
% \documentclass[10pt,a4paper,academicons]{altacv}
\documentclass[10pt,a4paper]{altacv}

%% AltaCV uses the fontawesome and academicon fonts
%% and packages. 
%% See texdoc.net/pkg/fontawecome and http://texdoc.net/pkg/academicons for full list of symbols.
%% When using the "academicons" option,
%% Compile with LuaLaTeX for best results. If you
%% want to use XeLaTeX, you may need to install
%% Academicons.ttf in your operating system's font %% folder.


% Change the page layout if you need to
\geometry{left=1cm,right=9cm,marginparwidth=6.8cm,marginparsep=1.2cm,top=1cm,bottom=1cm}

% Change the font if you want to.

% If using pdflatex:
\usepackage[utf8]{inputenc}
\usepackage[T1]{fontenc}
\usepackage[default]{lato}

% If using xelatex or lualatex:
% \setmainfont{Lato}

% Change the colours if you want to
\definecolor{VividPurple}{HTML}{2979ff}
\definecolor{SlateGrey}{HTML}{2E2E2E}
\definecolor{LightGrey}{HTML}{666666}
\colorlet{heading}{VividPurple}
\colorlet{accent}{VividPurple}
\colorlet{emphasis}{SlateGrey}
\colorlet{body}{LightGrey}

% Change the bullets for itemize and rating marker
% for \cvskill if you want to
\renewcommand{\itemmarker}{{\small\textbullet}}
\renewcommand{\ratingmarker}{\faCircle}

\begin{document}
\name{Lyam Milbury}
\tagline{COMPUTER ENGINEERING STUDENT | HARDWARE HACKER | TECH ENTHUSIAST}
\title{Lyam Milbury's Resume}
\personalinfo{%
  % Not all of these are required!
  % You can add your own with \printinfo{symbol}{detail}
  \normalsize\email{lyammil@outlook.com}
%  \mailaddress{Address, Street, 00000 County}
%  \location{Sunnyvale, CA}
  \normalsize\github{Lyam-Milbury}
  \normalsize\linkedin{Lyam-Milbury}
%  \divider
%  \divider
}

%% Make the header extend all the way to the right, if you want. Extend the right margin by 8cm (=6.8cm marginparwidth + 1.2cm marginparsep)
\begin{adjustwidth}{}{-8cm}
\makecvheader
\end{adjustwidth}

%% Provide the file name containing the sidebar contents as an optional parameter to \cvsection.
%% You can always just use \marginpar{...} if you do
%% not need to align the top of the contents to any
%% \cvsection title in the "main" bar.

\cvsection[page1sidebar]{experience}
\cvproject
	{Front End Developer}
    {Government of Canada}
    {May 2018 - Present}
\begin{itemize}
	\item {Developed and maintained webpages and tools for the Spectrum Telecommunication Services (STS) branch, working primarily on creating complex interactive google maps.}
	\item {Harnessed geoJSON, KMLs, and JavaScript to lead an initiative to replace over five hundred outdated spectrum auction webpages into a handful of interactive maps, reducing database ROT and client frustration. Received Award of Merit from the Deputy Minister of the STS for this work.}
	\item {Worked with a passionate team in a collaborative environment, focusing on teamwork in order to meet time-sensitive deadlines.}
	\item {Utilized downtime to create a Java-based application for quick and simple generation of interactive google maps to reduce webpage turnaround time.}
\end{itemize}

\cvsection[]{projects}

\cvproject
	{Applying ML for TCP}
    {Capstone Project}
    {Fall 2021 - Winter 2022}
\begin{itemize}
	\item {Collaborated with a professor and two peers to conduct research, train, and evaluate different ranking models utilizing test case prioritization (TCP) in the context of continuous integration (CI).}
	\item {Main responsibility was conducting research on different machine learning (ML) anomaly detection models, utilizing Python 3 and the Pyod toolkit to train/evaluate and optimize the selected models, and analyzing/visualizing their results.}
	\item {Also took on the role of data analyst, performing feature selection on large data sets from open source projects.}
\end{itemize}

\divider

\cvproject
	{catJAM - Discord music bot}
    {Personal Project}
    {September 2021}
\begin{itemize}
	\item {Developed a bot that can play high-quality audio from web-based mp4 videos in voice calls via user requests by leveraging NodeJS.}
	\item {Created with scalability and security in mind with multiple instances being active in different voice calls simultaneously.}
	\item {Hosted on an Ubuntu AWS EC2 instance to ensure a near 100\% uptime.}
\end{itemize}

\divider

\cvproject
	{Autonomous Greenhouse}
    {Course Project}
    {Fall 2020}
\begin{itemize}
	\item {Designed and built a prototype of an autonomous greenhouse that maintained optimal humidity, brightness, soil moisture and air quality levels with a team of two peers.}
	\item {Using Python 3, managed air quality control and ventilation. Converted sensor data into human-readable data, provided and managed an SQL database for sensor-to-sensor communication and designed/programmed a tkinter-based GUI for the database.}
	\item {Followed a realistic development cycle, including a design phase using finite state machines, use case diagrams and class diagrams.}
\end{itemize}


%% If the NEXT page doesn't start with a \cvsection but you'd
%% still like to add a sidebar, then use this command on THIS
%% page to add it. The optional argument lets you pull up the 
%% sidebar a bit so that it looks aligned with the top of the
%% main column.
% \addnextpagesidebar[-1ex]{page3sidebar}


\end{document}